\Section{Known Bugs}
\begin{description}
\item[ASCII-Ausgabe] Teile des Zeichens an Position 0 werden in den
darunterliegenden Zeilen ebenfalls angezeigt.
\item[Leitwerk, MMU] Beim LOAD-Befehl sollte die MMU lediglich die ben\"otigte
Anzahl an Bits laden und nicht immer 32 Bits.
\item[Leitwerk,MMU] Der Prozessor st\"urtzt, vermutlich wegen
Synchronisationsfehlern zwischen Leitwerk und MMU, gelegentlich ab. Au\ss{}erdem f\"uhrt ein Reset im schnellen Modus dadurch gelegentlich ebenfalls zu Inkonsistenzen und Programmabst\"urzen.
\item[ALU] Die ALU wartet wahrscheinlich l\"anger auf das Ergebnis der
Divisionseinheit, als n\"otig w\"are.
\item[MMU] Durch eine Unterscheidung der Speicherarten bei Speicherzugriffen
k\"onnte man hier eine Geschwindigkeitssteigerung erzielen.
\item[Leitwerk] Das Leitwerk wartet am Ende einiger Befehle noch 3 Takte auf
die ALU, ohne dass diese vorher gestartet wurde.
\item[MMU, allgemein] Die Buttons und Switches werden nicht entprellt.
\item[UART] Die UART-Einheit korrigiert keine Fehler in der \"Ubertragung,
weshalb nur eine sehr begrenzte Datenmenge fehlerfrei \"ubertragen werden kann.
Dadurch k\"onnen bei einer Programmierung durch die serielle Schnittstelle nur
sehr kleine Programme zum Einsatz kommen.
\item[UART] Die serielle Schnittstelle kann w\"ahrend der normalen
Codeausf\"uhrung des Prozessors nicht zuverl\"assig genutzt werden, da die
Refresh-Zyklen des DDR2-RAMs die Befehlsausf\"uhrung behindern und dadurch
\"ubertragene Bytes verpasst werden.
\item[allgemein] Ein Reset ist nur im langsamen Modus zuverl\"assig m\"oglich. Im
schnellen Modus kann eine Reset zum Absturz des Prozessors f\"uhren.
\end{description}
